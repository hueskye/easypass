\documentclass[a4paper,twocolumn,dvipdfm]{article}

\usepackage[utf8]{inputenc}
\usepackage[croatian]{babel}

\makeatletter
\renewcommand\thesection{\@arabic\c@section.}
\renewcommand\thesubsection{\thesection\@arabic\c@subsection.}
\renewcommand\thesubsubsection{\thesubsection\@arabic\c@subsubsection.}
\renewcommand\theequation{\@arabic\c@equation}
\renewcommand\thefigure{\@arabic\c@figure.}
\renewcommand\thetable{\@arabic\c@table.}
\makeatother

\usepackage{graphicx}
\usepackage{amssymb}
\usepackage{amsmath}
\usepackage{lmodern}
\usepackage[T1]{fontenc}

\begin{document}

\title{Generator pametnih \v{s}ifri}
\author{Viktor Braut, Matija Osre\v{c}ki i Dino Suli\'c}
\maketitle

\section*{Sa\v{z}etak}

Za ve\'cinu korisnika sigurnost na ra\v{c}unalnim sustavima bazira se na
kvalitetnoj \v{s}ifri. Ovaj rad predstavlja pristup generiranju kvalitetnih
\v{s}ifri na temelju lako\'ce utipkvanja istih. Za analizu lako\'ce utipkavanja
nau\v{c}ili smo neuronsku mre\v{z}u da regresijom ocjeni lako\'cu utipkavanja
nizova znakova fiksne duljine (engl., \emph{n-gram}), u na\v{s}em slu\v{c}aju
veli\v{c}ine 3. Optimalne parametre neuronske mre\v{z}e odredili smo unakrsnom
validacijom nad skupom od oko 440 podnizova koriste\'ci 4 preklopa. Iako
rezultati jako variraju, u kona\v{c}nici smo dobili dovoljno dobre rezultate za
izradu generatora. Generator koristi jednostavnu tehniku gdje kre\'ce s
slu\v{c}ajnom odabranom kratkom \v{s}ifrom dobre kvalitete i svaki sljede\'ci znak bira
stohati\v{c}ki na temelju kvalitete novodobivenog zadnjeg podniza.

\section{Uvod}

Sigurnost privatnosti na ra\v{c}unalnim sustavima danas je bitna vi\v{s}e nego
ikada. Za ve\'cinu korisnika to zna\v{c}i jednu stvar -- kvalitetna \v{s}ifra.
I dok sustavi poput GMail-a ugra\dj uju metode dodatne verifikacije korisnika
putem tokena generiranih primjerice mobilnim ure\dj ajem te postoje
rje\v{s}enja sigurnosti uporabom kriptografskih metoda javnih i privatnih
klju\v{c}eva (SSH, GPG, itd.), za ve\'cinu web servisa, operacijskih sustava i
ostalih oblika programske potpore kvalitetna \v{s}ifra je najzastupljenije
rje\v{s}enje. 

Vi\v{s}e je svojstava kvalitetne \v{s}ifre. Najbitnije svojstvo je sigurnost --
mora biti dovoljne duljine, sadr\v{z}avati dovoljno razli\v{c}itih vrsta
znakova (mala i velika slova, brojevi i ostali znakovi) koji bi trebali biti
slu\v{c}ajno raspore\dj eni, kako pojedini dijelovi \v{s}ifre ne bi bile
konkretne rije\v{c}i. Drugo po\v{z}eljno svojstvo je da je \v{s}ifru lagano
zapamtiti. Na\v{z}alost, za ve\'cinu je to najbitnije svojstvo zbog \v{c}ega za
\v{s}ifre koriste imena svojih djevojki, velikih kantautora (Bob) i sli\v{c}no.

U ovom radu predstavljamo ideju izbora, odnosno generiranja \v{s}ifre na
temelju lako\'ce utipkavanja iste. Prva pretpostavka je da \'ce takav na\v{c}in
zastupati sve znakove na tipkovnici u podjednakoj mjeri te da zbog slu\v{c}ajne
prirode generiranja \v{s}ifri ne\'ce do\'ci do podnizova koji se mogu na\'ci u
rje\v{c}nicima, prema tome bi trebalo biti zadovoljeno svojstvo sigurnosti.
Druga pretpostavka je da lako\'ca utipkavanja olak\v{s}ava mehani\v{c}ko
pam\'cenje i da \'ce time biti zadovoljeno drugo stvojstvo dobre \v{s}ifre,
iako mo\v{z}da tek nakon kra\'ceg perioda uvje\v{z}bavanja \v{s}ifre.

U\v{z}i aspekt ovog zadatka koji je i ujedno najte\v{z}i jest analiza lako\'ce
utipkavanja proizvoljnih nizova znakova na tipkovnici odre\dj enog rasporeda
znakova. Pristup koji smo prirodno prihvatili jest uporaba subjektivnih ocjena
skupa kra\'cih nizova u nadi da postoje nekakva statisti\v{c}ka ili geometrijska
korelacija izme\dj u transformiranih nizova znakova i na\v{s}ih subjektivnih ocjena.
Konkretno, koristili smo neuronske mre\v{z}e kako bi regresijom odredili lako\'ce
utipkvanja nizova znakova koje nismo vidjeli.

\section{Metode}

Po\v{s}to razli\v{c}iti rasporedi tipki utje\v{c}u na lako\'cu utipkavanja,
kori\v{s}ten je US raspored tipki. Tako\dj er ne koristimo znakove za koje je
potrebna tipka \texttt{shift} i podrazumijeva se da nitko ne koristi
\texttt{capslock}. Prema tome, sva slova su mala i koristi se podskup znakova.
Znakovi \texttt{\textbackslash} i \texttt{|} su na US tipkovnici na tipki koja
je ponekad iznad, a ponekad lijevo od tipke \texttt{enter}, zbog \v{c}ega
tako\dj er nisu kori\v{s}tene.

\subsection{Analiza lako\'ce tipkanja}

\subsubsection{Tehnika n-grama}

Pretpostavka koju koristimo prilikom analize lako\'ce utipkavanja proizvoljnog
teksta jest da ako uzmemo sve mogu\'ce podnizove fiksne duljine (n-grame),
lako\'ca utipkavanja tog niza znakova odgovara prosjeku ocjena lako\'ca
utipkavanja svih njegovih n-grama. Ta se pretpostavka temelji na vi\v{s}e
intuitivnih ideja. Prva je da ako su n-grami dovoljno dugi i dovoljno
kvalitetni, ono \v{s}to je trebalo utipkati prije $n$ znakova nije toliko
bitno, tj.\ toliko daleka povijest nema utjecaj na ono \v{s}to slijedi. Druga
ideja je da prilikom u\v{c}enja utipkavanja neke \v{s}ifre, korisnik \'ce
prirodno grupirati slova u manje grupe veli\v{c}ine 3 do 5, koje \'ce mo\'ci
instantno utipkati ako su kvalitetne. Time ujedno olak\v{s}avamo
ozna\v{c}avanje primjera, smanjujemo broj ulaznih kombinacija i omogu\'cujemo
u\v{c}enje neuronskim mre\v{z}om.

\subsubsection{Generiranje zna\v{c}ajki}

Za svaki znak generiramo 8 zna\v{c}ajki, koji se mogu podijeliti u tri grupe:
\begin{itemize}
        \item Koordinate trenutne tipke (2)
        \item Polarne koordinate vektora od prethodne do trenutne tipke (2)
        \item Polarne koordinate vektora od lijeve i desne tipke \texttt{alt}
            do trenutne tipke (4)
\end{itemize}
Pritom se vrijednosti svih zna\v{c}ajki normaliziraju na interval $[-1, 1]$.

Koordinate su temelj svih ovih zna\v{c}ajki. Koordinatni sustav je postavljen
sa $(0,0)$ na tipki $1$. Svaka tipka je kvadrat veli\v{c}ine $1$ sa $1$ i udaljenost izme\dj u 
tipki je $0.2$.

\subsection{Stohasti\v{c}ki generator \v{s}ifri}

\section{Rezultati}

\section{Zaklju\v{c}ak}

\section*{Literatuta}

Ovo smo radili isklju\v{c}ivo prema vlastitom naho\dj enju. Na internetu
je te\v{s}ko na\'ci materijale na ovu temu. Kao pomo\'c, koristili smo
isklju\v{c}ivo dokumentacije programskih biblioteka koje smo koristili
(\texttt{neurolab}, \texttt{numpy}).

\subsection*{Dodatak A: programsko ostvarenje}

\end{document}
